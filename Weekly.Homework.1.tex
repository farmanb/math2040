\documentclass{article}
\usepackage[utf8]{inputenc}
\usepackage[margin=1in,
    top=1in,
    bottom=1in]{geometry}
\usepackage{amsmath}
\usepackage{amsthm}
\usepackage{amssymb}
\usepackage{enumerate}

\newcommand{\Z}{\mathbb{Z}}
\newtheorem{theorem}{Theorem}
\numberwithin{theorem}{section}

\title{Weekly Homework 1}
\author{Blake Farman}
\date{January 2023}

\begin{document}

\maketitle
\setcounter{section}{1}
\section{Mathematics and Logic}
\subsection{A Taste of Number Theory}

\setcounter{theorem}{1}
\begin{theorem}
If \(n\) is an even integer, then \(n^2\) is an even integer.
\end{theorem}

\begin{proof}
%% Your proof goes here
\end{proof}

\begin{theorem}
The sum of two consecutive integers is odd.
\end{theorem}

\begin{proof}
%% Your proof goes here
\end{proof}

\setcounter{theorem}{6}
\begin{theorem}
The sum of any three consecutive integers is always divisible by three.
\end{theorem}

\begin{proof}
%% Your proof goes here
\end{proof}

\setcounter{theorem}{9}
\begin{theorem}
If \(a,n \in \Z\) such that \(a\) divides \(n\), then a divides \(-n\).
\end{theorem}

\begin{proof}
%% Your proof goes here
\end{proof}

\begin{theorem}
If \(a,n,m \in \Z\) such that \(a\) divides \(m\) and \(a\) divides \(n\), then \(a\) divides \(m+n\).
\end{theorem}

\begin{proof}
%% Your proof goes here
\end{proof}
\end{document}
